\documentclass[11pt,a4paper]{article}

% \XeTeX command
\usepackage{metalogo}

% Make frame visible (only for debug purposes)
\usepackage{showframe}

% Emulate MS Word
\usepackage{wordlike}
\usepackage[none]{hyphenat} % Word does not automatically hyphenate
\def\hyph{-\penalty0\hskip0pt\relax} % Word breaks words with hyphens automatically, LaTeX does not; define \hyph to achieve this behaviour in LaTeX as well; use '\hyph{}' to replace hyphens in words if you want LaTeX to break at this hyphen
\frenchspacing % No additional spacing after end of sentence

% One inch margins
%\PassOptionsToPackage{margin=1in,a4paper}{geometry}

% Font selection
\usepackage{fontspec}
\setmainfont[Mapping=tex-text]{Times New Roman}

% Remove footnote indentation
\usepackage[hang,flushmargin]{footmisc} 

% Redefine lists
\renewcommand{\labelitemi}{\raisebox{-0.42ex}{\LARGE{•}}}
\newfontfamily\myfont{Courier New} % Needed for bullet point of second level
\renewcommand{\labelitemii}{{\raisebox{0.23ex}{{\myfont{o}}}}\hspace{-0.3mm}} % raisebox makes it look better than the original
\renewcommand{\labelitemiii}{\raisebox{-0.15ex}{\Large{▪}}\hspace{0.15mm}}
\renewcommand{\labelitemiv}{\raisebox{-0.42ex}{\LARGE{•}}}
\let\olditemize\itemize
\renewcommand{\itemize}{
	\setlength{\leftmargini}{1.27cm} % Must be placed in preamble and outside list environment
	\setlength{\leftmarginii}{1.27cm} % Must be placed in preamble and outside list environment
	\setlength{\leftmarginiii}{1.27cm} % Must be placed in preamble and outside list environment
	\setlength{\leftmarginiv}{1.27cm} % Must be placed in preamble and outside list environment
	\olditemize
	\setlength{\topsep}{0pt}
	\setlength{\partopsep}{0pt}
	\setlength{\parskip}{0pt}
	\setlength{\itemsep}{0pt}
	\setlength{\parskip}{0pt}
	\setlength{\parsep}{0pt}
	\setlength{\labelsep}{0.44cm}
	\setlength{\labelwidth}{0.14cm}
}

\renewcommand{\labelenumii}{\arabic{enumi}.\arabic{enumii}}
\let\oldenumerate\enumerate
\renewcommand{\enumerate}{
	\setlength{\leftmargini}{0.63cm}
	\oldenumerate
	\setlength{\topsep}{0pt}
	\setlength{\partopsep}{0pt}
	\setlength{\parskip}{0pt}
	\setlength{\itemsep}{0pt}
	\setlength{\parskip}{0pt}
	\setlength{\parsep}{0pt}
	\setlength{\labelsep}{0.35cm}
	\setlength{\labelwidth}{0.14cm}
}

%\usepackage{enumitem} % Necessary to have labels of enums left-aligned. Must appear _after_ own list redefinitions!

\begin{document}
% 17 pt line spacing, must be inside \begin{document}
\setlength{\baselineskip}{17pt}
\fontdimen2\font=0.64ex% inter word space

\section{\XeTeX{} Rendering of Unordered Lists}
\begin{itemize}
	\item Auf der Registerkarte 'Einfügen' enthalten die Kataloge Elemente, die mit dem generellen Layout des Dokuments koordiniert werden sollten.
	\item Mithilfe dieser Kataloge können Sie Tabellen, Kopfzeilen, Fußzeilen, Listen, Deckblätter und sonstige Dokumentbausteine einfügen.
	\begin{itemize}
		\item Wenn Sie Bilder, Tabellen oder Diagramme erstellen, werden diese auch mit dem aktuellen Dokumentlayout koordiniert.
		\item Die Formatierung von markiertem Text im Dokumenttext kann auf einfache Weise geändert werden, indem Sie im Schnellformatvorlagen-Katalog auf der Registerkarte 'Start' ein Layout für den markierten Text auswählen.
		\begin{itemize}
			\item Text können Sie auch direkt mithilfe der anderen Steuerelemente auf der Registerkarte 'Start' formatieren.
			\item Die meisten Steuerelemente ermöglichen die Auswahl zwischen dem Layout des aktuellen Designs oder der direkten Angabe eines Formats.
			\begin{itemize}
				\item Wählen Sie neue Designelemente auf der Registerkarte 'Seitenlayout' aus, um das generelle Layout des Dokuments zu ändern.
				\item Verwenden Sie den Befehl zum Ändern des aktuellen Schnellformatvorlagen-Satzes, um die im Schnellformatvorlagen\hyph{}Katalog verfügbaren Formatvorlagen zu ändern.
			\end{itemize}
		\end{itemize}
	\end{itemize}
\end{itemize}
% Since LaTeX knows 'only' four levels of lists, we stop after four

\section{\XeTeX{} Rendering of Ordered Lists}
This is a paragraph of text.
\begin{enumerate}
	\item Mithilfe dieser Kataloge können Sie Tabellen, Kopfzeilen, Fußzeilen, Listen, Deckblätter und sonstige Dokumentbausteine einfügen.
	\item Wenn Sie Bilder, Tabellen oder Diagramme erstellen, werden diese auch mit dem aktuellen Dokumentlayout koordiniert.
	\begin{enumerate}
		\item Die Formatierung von markiertem Text im Dokumenttext kann auf einfache Weise geändert werden, indem Sie im Schnellformatvorlagen-Katalog auf der Registerkarte 'Start' ein Layout für den markierten Text auswählen.
		\item Text können Sie auch direkt mithilfe der anderen Steuerelemente auf der Registerkarte 'Start' formatieren.
		\begin{enumerate}
			\item Die meisten Steuerelemente ermöglichen die Auswahl zwischen dem Layout des aktuellen Designs oder der direkten Angabe eines Formats.
			\item Auf der Registerkarte 'Einfügen' enthalten die Kataloge Elemente, die mit dem generellen Layout des Dokuments koordiniert werden sollten.
			\begin{enumerate}
				\item Mithilfe dieser Kataloge können Sie Tabellen, Kopfzeilen, Fußzeilen, Listen, Deckblätter und sonstige Dokumentbausteine einfügen.
				\item Wenn Sie Bilder, Tabellen oder Diagramme erstellen, werden diese auch mit dem aktuellen Dokumentlayout koordiniert.
			\end{enumerate}
		\end{enumerate}
	\end{enumerate}
\end{enumerate}
\end{document}
